\documentclass[a4paper, 12pt]{report}

\title{Editeur web}
\author{Pierre Burc, Olivier Duplouy, Hamza Erraji, Issame Amal, Mickaël Berger, Joachim Divet, Zaydane Sadiki & Abdelhamid Belarbi}

%% Pour des marges plus équitables.
\usepackage[margin=1.5cm]{geometry}
%% Pour la langue des titres et sous-titres.
\usepackage[francais]{babel}
%% Pour de belles images.
\usepackage{graphicx}
%% Pour la police de caractères.
\usepackage{fontspec}
\setmainfont{Delicious-Roman}
%% Pour faire le glossaire.
\usepackage{glossaries}
\makeglossaries
%% Après première compilation écrire ligne suivante dans un terminal:
%% miktex-makeindex -s rapportStageAbdel.ist -t rapportStageAbdel.glg -o rapportStageAbdel.gls rapportStageAbdel.glo
%% re-compiler ensuite.
\begin{document}
	%% Glossaire
	\newglossaryentry{UML}
	{
		name=UML,
		description={Unified Modeling Language, langage de modélisation graphique à base de pictogrammes}
	}

	\newglossaryentry{diagramme de Gantt}
	{
		name=diagramme de Gantt,
		description={Le diagramme de Gantt est un outil utilisé en ordonnancement et gestion de projet et permettant de visualiser dans le temps
		les diverses tâches liées composant un projet. Il permet de représenter graphiquement l'avancement du projet}
	}
	
	\newglossaryentry{bogue}
	{
		name=bogue,
		description={En informatique, un bug (de l’anglais bug, « insecte ») ou bogue est un défaut de conception d'un programme
		informatique à l'origine d'un dysfonctionnement},
		plural=bogues
	}
	
	\begin{titlepage}
		\center{\includegraphics[width=5cm]{images/logoUM2.png}}	\\ 
		~\\
		~\\
		~\\
		~\\
		~\\		
		\begin{center}
			{\large Rapport préliminaire de projet} \\
			{\large Licence 3}\\
			\vspace{1,5cm}
			{\Huge Editeur de sites web}\\
			~\\
			~\\
			~\\
			\includegraphics[width=12.5cm]{images/logoTest1.png}
			~\\
			~\\
			{\large Réalisé par :} \\
			~\\
			{\LARGE Pierre Burc, Olivier Duplouy, \\
				      Hamza Erraji, Issame Amal,\\
				      Mickaël Berger, Joachim Divet,\\
				      Zaydane Sadiki et Abdelhamid Belarbi}\\
			\vspace{1,5cm}
			{\large Sous la direction de :} \\
			~\\
			{\LARGE Michel Meynard} \\
			\vspace{2.5cm}
			{\large Année universitaire 2011-2012}			
		\end{center}
	\end{titlepage}
%%%%%%%%%%%%%%%%%%%%%%%%%%%%%%%%%%%%%%%%
%%%%%%%%%%%%%%%%%%%%%%%%%%%%%%%%%%%%%%%%
	\begin{chapter}*{Remerciements}
	Merci
	\end{chapter}
%%%%%%%%%%%%%%%%%%%%%%%%%%%%%%%%%%%%%%%%
%%%%%%%%%%%%%%%%%%%%%%%%%%%%%%%%%%%%%%%%
	%% Table des matières.
	\tableofcontents
%%%%%%%%%%%%%%%%%%%%%%%%%%%%%%%%%%%%%%%%
%%%%%%%%%%%%%%%%%%%%%%%%%%%%%%%%%%%%%%%%
	\begin{chapter}*{Introduction}
	\addcontentsline{toc}{chapter}{Introduction}
	Petite description: lieu date contexte météo...
	\end{chapter}
%%%%%%%%%%%%%%%%%%%%%%%%%%%%%%%%%%%%%%%%
%%%%%%%%%%%%%%%%%%%%%%%%%%%%%%%%%%%%%%%%
	\begin{part}{Analyse}
		\begin{chapter}{Cahier des charges}
		\end{chapter}
		\begin{chapter}{Étude de projets existants}
		\end{chapter}
		\begin{chapter}{Choix des outils}
		\end{chapter}
		\begin{chapter}{Organisation}
		\end{chapter}
	\end{part}
%%%%%%%%%%%%%%%%%%%%%%%%%%%%%%%%%%%%%%%%
%%%%%%%%%%%%%%%%%%%%%%%%%%%%%%%%%%%%%%%%
	\begin{part}{Conception}
		\begin{chapter}{Décomposition en sous systèmes}
					\gls{UML}
		\end{chapter}
		\begin{chapter}{Cas d'utilisations}
		\end{chapter}
		\begin{chapter}{Diagrammes des classes}
		\end{chapter}
		\begin{chapter}{Diagrammes d'états-transitions}
			Pas sûr on verra si on à le temps.
		\end{chapter}
	\end{part}
%%%%%%%%%%%%%%%%%%%%%%%%%%%%%%%%%%%%%%%%
%%%%%%%%%%%%%%%%%%%%%%%%%%%%%%%%%%%%%%%%
	\begin{part}{L'oeuvre}
		\begin{chapter}{Travail de groupe}
			Là on parle des réunions, pourquoi pas mettre un extrait de journal.
		\end{chapter}
		\begin{chapter}{Implémentation}
			Ici on aborde les problèmes de la documentation, de nouvelles bibliothèques, de mettre en place nos algorithmes "en vrai", ... 
		\end{chapter}
		\begin{chapter}{Résultat}
			On va essayer de caser des trucs ici.
		\end{chapter}
		\begin{chapter}{Discussion}
			Critique (positive) du résultat, le cahier des charges est-il respecté ? améliorations possibles, erreurs qu'on à pu faire.
		\end{chapter}
	\end{part}
%%%%%%%%%%%%%%%%%%%%%%%%%%%%%%%%%%%%%%%%
%%%%%%%%%%%%%%%%%%%%%%%%%%%%%%%%%%%%%%%%
	\begin{chapter}*{Conclusion}
		Oué on s'est bien marré et tout et tout.
	\end{chapter}
%%%%%%%%%%%%%%%%%%%%%%%%%%%%%%%%%%%%%%%%
%%%%%%%%%%%%%%%%%%%%%%%%%%%%%%%%%%%%%%%%
	%% Glossaire
	\renewcommand\glossaryname{Glossaire}
	\printglossaries
	%% Table des figures
	\listoffigures
%%%%%%%%%%%%%%%%%%%%%%%%%%%%%%%%%%%%%%%%
%%%%%%%%%%%%%%%%%%%%%%%%%%%%%%%%%%%%%%%%
	%% Sitographie
	\renewcommand\bibname{Sitographie}%% Changement du titre de bibliographie en sitographie.
	\begin{thebibliography}{2}
		\bibitem{wikipedia}
		Wikipédia : http://fr.wikipedia.org/wiki\\
		L'encyclopédie en ligne de laquelle j'ai tiré certaines définitions présentes dans le glossaire.
		~\\
		\bibitem{yUML}
		yUML : www.yuml.me \\
		Ce site permet de générer à la volée des diagrammes \gls{UML}, extrêmement utile lorsqu'il s'agit de travail de groupe.
	\end{thebibliography}
\end{document}
